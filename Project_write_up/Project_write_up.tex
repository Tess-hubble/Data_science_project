\documentclass[11pt,preprint, authoryear]{elsarticle}

\usepackage{lmodern}
%%%% My spacing
\usepackage{setspace}
\setstretch{1.2}
\DeclareMathSizes{12}{14}{10}{10}

% Wrap around which gives all figures included the [H] command, or places it "here". This can be tedious to code in Rmarkdown.
\usepackage{float}
\let\origfigure\figure
\let\endorigfigure\endfigure
\renewenvironment{figure}[1][2] {
    \expandafter\origfigure\expandafter[H]
} {
    \endorigfigure
}

\let\origtable\table
\let\endorigtable\endtable
\renewenvironment{table}[1][2] {
    \expandafter\origtable\expandafter[H]
} {
    \endorigtable
}


\usepackage{ifxetex,ifluatex}
\usepackage{fixltx2e} % provides \textsubscript
\ifnum 0\ifxetex 1\fi\ifluatex 1\fi=0 % if pdftex
  \usepackage[T1]{fontenc}
  \usepackage[utf8]{inputenc}
\else % if luatex or xelatex
  \ifxetex
    \usepackage{mathspec}
    \usepackage{xltxtra,xunicode}
  \else
    \usepackage{fontspec}
  \fi
  \defaultfontfeatures{Mapping=tex-text,Scale=MatchLowercase}
  \newcommand{\euro}{€}
\fi

\usepackage{amssymb, amsmath, amsthm, amsfonts}

\def\bibsection{\section*{References}} %%% Make "References" appear before bibliography


\usepackage[round]{natbib}

\usepackage{longtable}
\usepackage[margin=2.3cm,bottom=2cm,top=2.5cm, includefoot]{geometry}
\usepackage{fancyhdr}
\usepackage[bottom, hang, flushmargin]{footmisc}
\usepackage{graphicx}
\numberwithin{equation}{section}
\numberwithin{figure}{section}
\numberwithin{table}{section}
\setlength{\parindent}{0cm}
\setlength{\parskip}{1.3ex plus 0.5ex minus 0.3ex}
\usepackage{textcomp}
\renewcommand{\headrulewidth}{0.2pt}
\renewcommand{\footrulewidth}{0.3pt}

\usepackage{array}
\newcolumntype{x}[1]{>{\centering\arraybackslash\hspace{0pt}}p{#1}}

%%%%  Remove the "preprint submitted to" part. Don't worry about this either, it just looks better without it:
\makeatletter
\def\ps@pprintTitle{%
  \let\@oddhead\@empty
  \let\@evenhead\@empty
  \let\@oddfoot\@empty
  \let\@evenfoot\@oddfoot
}
\makeatother

 \def\tightlist{} % This allows for subbullets!

\usepackage{hyperref}
\hypersetup{breaklinks=true,
            bookmarks=true,
            colorlinks=true,
            citecolor=blue,
            urlcolor=blue,
            linkcolor=blue,
            pdfborder={0 0 0}}


% The following packages allow huxtable to work:
\usepackage{siunitx}
\usepackage{multirow}
\usepackage{hhline}
\usepackage{calc}
\usepackage{tabularx}
\usepackage{booktabs}
\usepackage{caption}


\newenvironment{columns}[1][]{}{}

\newenvironment{column}[1]{\begin{minipage}{#1}\ignorespaces}{%
\end{minipage}
\ifhmode\unskip\fi
\aftergroup\useignorespacesandallpars}

\def\useignorespacesandallpars#1\ignorespaces\fi{%
#1\fi\ignorespacesandallpars}

\makeatletter
\def\ignorespacesandallpars{%
  \@ifnextchar\par
    {\expandafter\ignorespacesandallpars\@gobble}%
    {}%
}
\makeatother

\newenvironment{CSLReferences}[2]{%
}

\urlstyle{same}  % don't use monospace font for urls
\setlength{\parindent}{0pt}
\setlength{\parskip}{6pt plus 2pt minus 1pt}
\setlength{\emergencystretch}{3em}  % prevent overfull lines
\setcounter{secnumdepth}{5}

%%% Use protect on footnotes to avoid problems with footnotes in titles
\let\rmarkdownfootnote\footnote%
\def\footnote{\protect\rmarkdownfootnote}
\IfFileExists{upquote.sty}{\usepackage{upquote}}{}

%%% Include extra packages specified by user
\usepackage{array}
\usepackage{caption}
\usepackage{graphicx}
\usepackage{siunitx}
\usepackage[normalem]{ulem}
\usepackage{colortbl}
\usepackage{multirow}
\usepackage{hhline}
\usepackage{calc}
\usepackage{tabularx}
\usepackage{threeparttable}
\usepackage{wrapfig}
\usepackage{adjustbox}
\usepackage{hyperref}

%%% Hard setting column skips for reports - this ensures greater consistency and control over the length settings in the document.
%% page layout
%% paragraphs
\setlength{\baselineskip}{12pt plus 0pt minus 0pt}
\setlength{\parskip}{12pt plus 0pt minus 0pt}
\setlength{\parindent}{0pt plus 0pt minus 0pt}
%% floats
\setlength{\floatsep}{12pt plus 0 pt minus 0pt}
\setlength{\textfloatsep}{20pt plus 0pt minus 0pt}
\setlength{\intextsep}{14pt plus 0pt minus 0pt}
\setlength{\dbltextfloatsep}{20pt plus 0pt minus 0pt}
\setlength{\dblfloatsep}{14pt plus 0pt minus 0pt}
%% maths
\setlength{\abovedisplayskip}{12pt plus 0pt minus 0pt}
\setlength{\belowdisplayskip}{12pt plus 0pt minus 0pt}
%% lists
\setlength{\topsep}{10pt plus 0pt minus 0pt}
\setlength{\partopsep}{3pt plus 0pt minus 0pt}
\setlength{\itemsep}{5pt plus 0pt minus 0pt}
\setlength{\labelsep}{8mm plus 0mm minus 0mm}
\setlength{\parsep}{\the\parskip}
\setlength{\listparindent}{\the\parindent}
%% verbatim
\setlength{\fboxsep}{5pt plus 0pt minus 0pt}



\begin{document}



\begin{frontmatter}  %

\title{Predicting Total Auction Values in the Cape Colony}

% Set to FALSE if wanting to remove title (for submission)




\author[Add1]{Tessa Hubble}
\ead{21559953@sun.ac.za}





\address[Add1]{Stellenbosch University, South Africa}


\begin{abstract}
\small{
Predicting total auction values from a Cape Colony auction dataset
between 1720 and 1820.
}
\end{abstract}

\vspace{1cm}





\vspace{0.5cm}

\end{frontmatter}

\setcounter{footnote}{0}



%________________________
% Header and Footers
%%%%%%%%%%%%%%%%%%%%%%%%%%%%%%%%%
\pagestyle{fancy}
\chead{}
\rhead{}
\lfoot{}
\rfoot{\footnotesize Page \thepage}
\lhead{}
%\rfoot{\footnotesize Page \thepage } % "e.g. Page 2"
\cfoot{}

%\setlength\headheight{30pt}
%%%%%%%%%%%%%%%%%%%%%%%%%%%%%%%%%
%________________________

\headsep 35pt % So that header does not go over title




\hypertarget{introduction}{%
\section{\texorpdfstring{Introduction
\label{Introduction}}{Introduction }}\label{introduction}}

Over the first 150 years of the Cape Colony, deceased estate auctions
were an important means of exchange for the burgeoning population. The
only other means of purchasing goods was through ships that docked along
the coast. From livestock and furniture to slaves, a wide variety of
goods were obtained through auctions. Although auction rolls from this
time period have been transcribed, inconsistent spelling and errors are
pervasive. 30 different types of goods have been identified and tagged
throughout the auction rolls. This project aims to use this subset of
goods within auctions to predict total auction values. Focusing on
approximately 1400 auctions between 1720 and 1820, two different machine
learning techniques will be used to predict total auction values. Being
able to assess the total auction value based on a subset of goods can be
an important tool within historical data where there is limited
information. Exploratory data analysis will provide context and support
for the features included in the models. Throughout the exploratory data
analysis and subsequent models, it becomes clear that the number of
slaves, the date and the auction size consistently place upward pressure
on total auction values. The performance of a k-nearest neighbours model
and random forest model are compared to that of a linear regression to
assess accuracy.Ultimately, it is concluded that the subset of goods
offer limited predictive power.

\newpage

\hypertarget{auction-data}{%
\section{Auction data}\label{auction-data}}

Table 1 below provides an example of a portion of sales from one auction
in 1770. This auction roll shows goods sold, the price and the type of
good sold. A buyer\_id column exists as well. Within this column, some
names include titles such as ``de Wede'' (widow) or ``mijnheer''
(mister) which offers the possibility to determine how many titled men
or women attend an auction. Table 1 shows that only a few goods are
tagged.There are 26 different tagged goods in total. The number of
different types of tagged goods per auction will act as predictors. It
is therefore possible that the accuracy of the machine learning models
included may be hampered by limited presence of these goods across
auctions.

 
  \providecommand{\huxb}[2]{\arrayrulecolor[RGB]{#1}\global\arrayrulewidth=#2pt}
  \providecommand{\huxvb}[2]{\color[RGB]{#1}\vrule width #2pt}
  \providecommand{\huxtpad}[1]{\rule{0pt}{#1}}
  \providecommand{\huxbpad}[1]{\rule[-#1]{0pt}{#1}}

\begin{table}[ht]
\begin{centerbox}
\begin{threeparttable}
 \label{tab:Table1}
\setlength{\tabcolsep}{0pt}
\begin{tabular}{l l l l l}


\hhline{>{\huxb{0, 0, 0}{0.4}}->{\huxb{0, 0, 0}{0.4}}->{\huxb{0, 0, 0}{0.4}}->{\huxb{0, 0, 0}{0.4}}->{\huxb{0, 0, 0}{0.4}}-}
\arrayrulecolor{black}

\multicolumn{1}{!{\huxvb{0, 0, 0}{0.4}}l!{\huxvb{0, 0, 0}{0}}}{\huxtpad{6pt + 1em}\raggedright \hspace{6pt} \textbf{mooc\_id} \hspace{6pt}\huxbpad{6pt}} &
\multicolumn{1}{l!{\huxvb{0, 0, 0}{0}}}{\huxtpad{6pt + 1em}\raggedright \hspace{6pt} \textbf{purchase\_id} \hspace{6pt}\huxbpad{6pt}} &
\multicolumn{1}{l!{\huxvb{0, 0, 0}{0}}}{\huxtpad{6pt + 1em}\raggedright \hspace{6pt} \textbf{price} \hspace{6pt}\huxbpad{6pt}} &
\multicolumn{1}{l!{\huxvb{0, 0, 0}{0}}}{\huxtpad{6pt + 1em}\raggedright \hspace{6pt} \textbf{goods} \hspace{6pt}\huxbpad{6pt}} &
\multicolumn{1}{l!{\huxvb{0, 0, 0}{0.4}}}{\huxtpad{6pt + 1em}\raggedright \hspace{6pt} \textbf{type} \hspace{6pt}\huxbpad{6pt}} \tabularnewline[-0.5pt]


\hhline{>{\huxb{0, 0, 0}{0.4}}->{\huxb{0, 0, 0}{0.4}}->{\huxb{0, 0, 0}{0.4}}->{\huxb{0, 0, 0}{0.4}}->{\huxb{0, 0, 0}{0.4}}-}
\arrayrulecolor{black}

\multicolumn{1}{!{\huxvb{0, 0, 0}{0.4}}l!{\huxvb{0, 0, 0}{0}}}{\cellcolor[RGB]{242, 242, 242}\huxtpad{6pt + 1em}\raggedright \hspace{6pt} MOOC10/10.1 \hspace{6pt}\huxbpad{6pt}} &
\multicolumn{1}{l!{\huxvb{0, 0, 0}{0}}}{\cellcolor[RGB]{242, 242, 242}\huxtpad{6pt + 1em}\raggedright \hspace{6pt} MOOC10/10.1/1 \hspace{6pt}\huxbpad{6pt}} &
\multicolumn{1}{l!{\huxvb{0, 0, 0}{0}}}{\cellcolor[RGB]{242, 242, 242}\huxtpad{6pt + 1em}\raggedright \hspace{6pt} 0.7 \hspace{6pt}\huxbpad{6pt}} &
\multicolumn{1}{l!{\huxvb{0, 0, 0}{0}}}{\cellcolor[RGB]{242, 242, 242}\huxtpad{6pt + 1em}\raggedright \hspace{6pt} 1 partij flessen, bottels en aardwerk \hspace{6pt}\huxbpad{6pt}} &
\multicolumn{1}{l!{\huxvb{0, 0, 0}{0.4}}}{\cellcolor[RGB]{242, 242, 242}\huxtpad{6pt + 1em}\raggedright \hspace{6pt}  \hspace{6pt}\huxbpad{6pt}} \tabularnewline[-0.5pt]


\hhline{>{\huxb{0, 0, 0}{0.4}}|>{\huxb{0, 0, 0}{0.4}}|}
\arrayrulecolor{black}

\multicolumn{1}{!{\huxvb{0, 0, 0}{0.4}}l!{\huxvb{0, 0, 0}{0}}}{\huxtpad{6pt + 1em}\raggedright \hspace{6pt} MOOC10/10.1 \hspace{6pt}\huxbpad{6pt}} &
\multicolumn{1}{l!{\huxvb{0, 0, 0}{0}}}{\huxtpad{6pt + 1em}\raggedright \hspace{6pt} MOOC10/10.1/10 \hspace{6pt}\huxbpad{6pt}} &
\multicolumn{1}{l!{\huxvb{0, 0, 0}{0}}}{\huxtpad{6pt + 1em}\raggedright \hspace{6pt} 2.4 \hspace{6pt}\huxbpad{6pt}} &
\multicolumn{1}{l!{\huxvb{0, 0, 0}{0}}}{\huxtpad{6pt + 1em}\raggedright \hspace{6pt} 2 steenen vaderl:s seep en 2 steenen Caapse seep \hspace{6pt}\huxbpad{6pt}} &
\multicolumn{1}{l!{\huxvb{0, 0, 0}{0.4}}}{\huxtpad{6pt + 1em}\raggedright \hspace{6pt}  \hspace{6pt}\huxbpad{6pt}} \tabularnewline[-0.5pt]


\hhline{>{\huxb{0, 0, 0}{0.4}}|>{\huxb{0, 0, 0}{0.4}}|}
\arrayrulecolor{black}

\multicolumn{1}{!{\huxvb{0, 0, 0}{0.4}}l!{\huxvb{0, 0, 0}{0}}}{\cellcolor[RGB]{242, 242, 242}\huxtpad{6pt + 1em}\raggedright \hspace{6pt} MOOC10/10.1 \hspace{6pt}\huxbpad{6pt}} &
\multicolumn{1}{l!{\huxvb{0, 0, 0}{0}}}{\cellcolor[RGB]{242, 242, 242}\huxtpad{6pt + 1em}\raggedright \hspace{6pt} MOOC10/10.1/11 \hspace{6pt}\huxbpad{6pt}} &
\multicolumn{1}{l!{\huxvb{0, 0, 0}{0}}}{\cellcolor[RGB]{242, 242, 242}\huxtpad{6pt + 1em}\raggedright \hspace{6pt} 2.1 \hspace{6pt}\huxbpad{6pt}} &
\multicolumn{1}{l!{\huxvb{0, 0, 0}{0}}}{\cellcolor[RGB]{242, 242, 242}\huxtpad{6pt + 1em}\raggedright \hspace{6pt} 1 casje met thee en 1 bijbel \hspace{6pt}\huxbpad{6pt}} &
\multicolumn{1}{l!{\huxvb{0, 0, 0}{0.4}}}{\cellcolor[RGB]{242, 242, 242}\huxtpad{6pt + 1em}\raggedright \hspace{6pt} books \hspace{6pt}\huxbpad{6pt}} \tabularnewline[-0.5pt]


\hhline{>{\huxb{0, 0, 0}{0.4}}|>{\huxb{0, 0, 0}{0.4}}|}
\arrayrulecolor{black}

\multicolumn{1}{!{\huxvb{0, 0, 0}{0.4}}l!{\huxvb{0, 0, 0}{0}}}{\huxtpad{6pt + 1em}\raggedright \hspace{6pt} MOOC10/10.1 \hspace{6pt}\huxbpad{6pt}} &
\multicolumn{1}{l!{\huxvb{0, 0, 0}{0}}}{\huxtpad{6pt + 1em}\raggedright \hspace{6pt} MOOC10/10.1/12 \hspace{6pt}\huxbpad{6pt}} &
\multicolumn{1}{l!{\huxvb{0, 0, 0}{0}}}{\huxtpad{6pt + 1em}\raggedright \hspace{6pt} 3.1 \hspace{6pt}\huxbpad{6pt}} &
\multicolumn{1}{l!{\huxvb{0, 0, 0}{0}}}{\huxtpad{6pt + 1em}\raggedright \hspace{6pt} 6 kussens \hspace{6pt}\huxbpad{6pt}} &
\multicolumn{1}{l!{\huxvb{0, 0, 0}{0.4}}}{\huxtpad{6pt + 1em}\raggedright \hspace{6pt}  \hspace{6pt}\huxbpad{6pt}} \tabularnewline[-0.5pt]


\hhline{>{\huxb{0, 0, 0}{0.4}}|>{\huxb{0, 0, 0}{0.4}}|}
\arrayrulecolor{black}

\multicolumn{1}{!{\huxvb{0, 0, 0}{0.4}}l!{\huxvb{0, 0, 0}{0}}}{\cellcolor[RGB]{242, 242, 242}\huxtpad{6pt + 1em}\raggedright \hspace{6pt} MOOC10/10.1 \hspace{6pt}\huxbpad{6pt}} &
\multicolumn{1}{l!{\huxvb{0, 0, 0}{0}}}{\cellcolor[RGB]{242, 242, 242}\huxtpad{6pt + 1em}\raggedright \hspace{6pt} MOOC10/10.1/13 \hspace{6pt}\huxbpad{6pt}} &
\multicolumn{1}{l!{\huxvb{0, 0, 0}{0}}}{\cellcolor[RGB]{242, 242, 242}\huxtpad{6pt + 1em}\raggedright \hspace{6pt} 3.2 \hspace{6pt}\huxbpad{6pt}} &
\multicolumn{1}{l!{\huxvb{0, 0, 0}{0}}}{\cellcolor[RGB]{242, 242, 242}\huxtpad{6pt + 1em}\raggedright \hspace{6pt} 5 kussens \hspace{6pt}\huxbpad{6pt}} &
\multicolumn{1}{l!{\huxvb{0, 0, 0}{0.4}}}{\cellcolor[RGB]{242, 242, 242}\huxtpad{6pt + 1em}\raggedright \hspace{6pt}  \hspace{6pt}\huxbpad{6pt}} \tabularnewline[-0.5pt]


\hhline{>{\huxb{0, 0, 0}{0.4}}|>{\huxb{0, 0, 0}{0.4}}|}
\arrayrulecolor{black}

\multicolumn{1}{!{\huxvb{0, 0, 0}{0.4}}l!{\huxvb{0, 0, 0}{0}}}{\huxtpad{6pt + 1em}\raggedright \hspace{6pt} MOOC10/10.1 \hspace{6pt}\huxbpad{6pt}} &
\multicolumn{1}{l!{\huxvb{0, 0, 0}{0}}}{\huxtpad{6pt + 1em}\raggedright \hspace{6pt} MOOC10/10.1/14 \hspace{6pt}\huxbpad{6pt}} &
\multicolumn{1}{l!{\huxvb{0, 0, 0}{0}}}{\huxtpad{6pt + 1em}\raggedright \hspace{6pt} 25.4 \hspace{6pt}\huxbpad{6pt}} &
\multicolumn{1}{l!{\huxvb{0, 0, 0}{0}}}{\huxtpad{6pt + 1em}\raggedright \hspace{6pt} 1 bed, 1 peul, 4 cussens en 2 combaarsen \hspace{6pt}\huxbpad{6pt}} &
\multicolumn{1}{l!{\huxvb{0, 0, 0}{0.4}}}{\huxtpad{6pt + 1em}\raggedright \hspace{6pt} beds \hspace{6pt}\huxbpad{6pt}} \tabularnewline[-0.5pt]


\hhline{>{\huxb{0, 0, 0}{0.4}}|>{\huxb{0, 0, 0}{0.4}}|}
\arrayrulecolor{black}

\multicolumn{1}{!{\huxvb{0, 0, 0}{0.4}}l!{\huxvb{0, 0, 0}{0}}}{\cellcolor[RGB]{242, 242, 242}\huxtpad{6pt + 1em}\raggedright \hspace{6pt} MOOC10/10.1 \hspace{6pt}\huxbpad{6pt}} &
\multicolumn{1}{l!{\huxvb{0, 0, 0}{0}}}{\cellcolor[RGB]{242, 242, 242}\huxtpad{6pt + 1em}\raggedright \hspace{6pt} MOOC10/10.1/15 \hspace{6pt}\huxbpad{6pt}} &
\multicolumn{1}{l!{\huxvb{0, 0, 0}{0}}}{\cellcolor[RGB]{242, 242, 242}\huxtpad{6pt + 1em}\raggedright \hspace{6pt} 5.2 \hspace{6pt}\huxbpad{6pt}} &
\multicolumn{1}{l!{\huxvb{0, 0, 0}{0}}}{\cellcolor[RGB]{242, 242, 242}\huxtpad{6pt + 1em}\raggedright \hspace{6pt} 1 bijbel en 1 nagtmaalboek met silvere beslag en 4 boeken \hspace{6pt}\huxbpad{6pt}} &
\multicolumn{1}{l!{\huxvb{0, 0, 0}{0.4}}}{\cellcolor[RGB]{242, 242, 242}\huxtpad{6pt + 1em}\raggedright \hspace{6pt} books \hspace{6pt}\huxbpad{6pt}} \tabularnewline[-0.5pt]


\hhline{>{\huxb{0, 0, 0}{0.4}}|>{\huxb{0, 0, 0}{0.4}}|}
\arrayrulecolor{black}

\multicolumn{1}{!{\huxvb{0, 0, 0}{0.4}}l!{\huxvb{0, 0, 0}{0}}}{\huxtpad{6pt + 1em}\raggedright \hspace{6pt} MOOC10/10.1 \hspace{6pt}\huxbpad{6pt}} &
\multicolumn{1}{l!{\huxvb{0, 0, 0}{0}}}{\huxtpad{6pt + 1em}\raggedright \hspace{6pt} MOOC10/10.1/15 \hspace{6pt}\huxbpad{6pt}} &
\multicolumn{1}{l!{\huxvb{0, 0, 0}{0}}}{\huxtpad{6pt + 1em}\raggedright \hspace{6pt} 5.2 \hspace{6pt}\huxbpad{6pt}} &
\multicolumn{1}{l!{\huxvb{0, 0, 0}{0}}}{\huxtpad{6pt + 1em}\raggedright \hspace{6pt} 1 bijbel en 1 nagtmaalboek met silvere beslag en 4 boeken \hspace{6pt}\huxbpad{6pt}} &
\multicolumn{1}{l!{\huxvb{0, 0, 0}{0.4}}}{\huxtpad{6pt + 1em}\raggedright \hspace{6pt} books \hspace{6pt}\huxbpad{6pt}} \tabularnewline[-0.5pt]


\hhline{>{\huxb{0, 0, 0}{0.4}}|>{\huxb{0, 0, 0}{0.4}}|}
\arrayrulecolor{black}

\multicolumn{1}{!{\huxvb{0, 0, 0}{0.4}}l!{\huxvb{0, 0, 0}{0}}}{\cellcolor[RGB]{242, 242, 242}\huxtpad{6pt + 1em}\raggedright \hspace{6pt} MOOC10/10.1 \hspace{6pt}\huxbpad{6pt}} &
\multicolumn{1}{l!{\huxvb{0, 0, 0}{0}}}{\cellcolor[RGB]{242, 242, 242}\huxtpad{6pt + 1em}\raggedright \hspace{6pt} MOOC10/10.1/16 \hspace{6pt}\huxbpad{6pt}} &
\multicolumn{1}{l!{\huxvb{0, 0, 0}{0}}}{\cellcolor[RGB]{242, 242, 242}\huxtpad{6pt + 1em}\raggedright \hspace{6pt} 0.5 \hspace{6pt}\huxbpad{6pt}} &
\multicolumn{1}{l!{\huxvb{0, 0, 0}{0}}}{\cellcolor[RGB]{242, 242, 242}\huxtpad{6pt + 1em}\raggedright \hspace{6pt} 1 partij olijtijten \hspace{6pt}\huxbpad{6pt}} &
\multicolumn{1}{l!{\huxvb{0, 0, 0}{0.4}}}{\cellcolor[RGB]{242, 242, 242}\huxtpad{6pt + 1em}\raggedright \hspace{6pt}  \hspace{6pt}\huxbpad{6pt}} \tabularnewline[-0.5pt]


\hhline{>{\huxb{0, 0, 0}{0.4}}|>{\huxb{0, 0, 0}{0.4}}|}
\arrayrulecolor{black}

\multicolumn{1}{!{\huxvb{0, 0, 0}{0.4}}l!{\huxvb{0, 0, 0}{0}}}{\huxtpad{6pt + 1em}\raggedright \hspace{6pt} MOOC10/10.1 \hspace{6pt}\huxbpad{6pt}} &
\multicolumn{1}{l!{\huxvb{0, 0, 0}{0}}}{\huxtpad{6pt + 1em}\raggedright \hspace{6pt} MOOC10/10.1/17 \hspace{6pt}\huxbpad{6pt}} &
\multicolumn{1}{l!{\huxvb{0, 0, 0}{0}}}{\huxtpad{6pt + 1em}\raggedright \hspace{6pt} 1.6 \hspace{6pt}\huxbpad{6pt}} &
\multicolumn{1}{l!{\huxvb{0, 0, 0}{0}}}{\huxtpad{6pt + 1em}\raggedright \hspace{6pt} 1 partij canten en lind \hspace{6pt}\huxbpad{6pt}} &
\multicolumn{1}{l!{\huxvb{0, 0, 0}{0.4}}}{\huxtpad{6pt + 1em}\raggedright \hspace{6pt}  \hspace{6pt}\huxbpad{6pt}} \tabularnewline[-0.5pt]


\hhline{>{\huxb{0, 0, 0}{0.4}}|>{\huxb{0, 0, 0}{0.4}}|}
\arrayrulecolor{black}

\multicolumn{1}{!{\huxvb{0, 0, 0}{0.4}}l!{\huxvb{0, 0, 0}{0}}}{\cellcolor[RGB]{242, 242, 242}\huxtpad{6pt + 1em}\raggedright \hspace{6pt} MOOC10/10.1 \hspace{6pt}\huxbpad{6pt}} &
\multicolumn{1}{l!{\huxvb{0, 0, 0}{0}}}{\cellcolor[RGB]{242, 242, 242}\huxtpad{6pt + 1em}\raggedright \hspace{6pt} MOOC10/10.1/18 \hspace{6pt}\huxbpad{6pt}} &
\multicolumn{1}{l!{\huxvb{0, 0, 0}{0}}}{\cellcolor[RGB]{242, 242, 242}\huxtpad{6pt + 1em}\raggedright \hspace{6pt} 9.4 \hspace{6pt}\huxbpad{6pt}} &
\multicolumn{1}{l!{\huxvb{0, 0, 0}{0}}}{\cellcolor[RGB]{242, 242, 242}\huxtpad{6pt + 1em}\raggedright \hspace{6pt} 2 swarte chitsen \hspace{6pt}\huxbpad{6pt}} &
\multicolumn{1}{l!{\huxvb{0, 0, 0}{0.4}}}{\cellcolor[RGB]{242, 242, 242}\huxtpad{6pt + 1em}\raggedright \hspace{6pt}  \hspace{6pt}\huxbpad{6pt}} \tabularnewline[-0.5pt]


\hhline{>{\huxb{0, 0, 0}{0.4}}|>{\huxb{0, 0, 0}{0.4}}|}
\arrayrulecolor{black}

\multicolumn{1}{!{\huxvb{0, 0, 0}{0.4}}l!{\huxvb{0, 0, 0}{0}}}{\huxtpad{6pt + 1em}\raggedright \hspace{6pt} MOOC10/10.1 \hspace{6pt}\huxbpad{6pt}} &
\multicolumn{1}{l!{\huxvb{0, 0, 0}{0}}}{\huxtpad{6pt + 1em}\raggedright \hspace{6pt} MOOC10/10.1/19 \hspace{6pt}\huxbpad{6pt}} &
\multicolumn{1}{l!{\huxvb{0, 0, 0}{0}}}{\huxtpad{6pt + 1em}\raggedright \hspace{6pt} 9.7 \hspace{6pt}\huxbpad{6pt}} &
\multicolumn{1}{l!{\huxvb{0, 0, 0}{0}}}{\huxtpad{6pt + 1em}\raggedright \hspace{6pt} 2 swarte chitsen \hspace{6pt}\huxbpad{6pt}} &
\multicolumn{1}{l!{\huxvb{0, 0, 0}{0.4}}}{\huxtpad{6pt + 1em}\raggedright \hspace{6pt}  \hspace{6pt}\huxbpad{6pt}} \tabularnewline[-0.5pt]


\hhline{>{\huxb{0, 0, 0}{0.4}}|>{\huxb{0, 0, 0}{0.4}}|}
\arrayrulecolor{black}

\multicolumn{1}{!{\huxvb{0, 0, 0}{0.4}}l!{\huxvb{0, 0, 0}{0}}}{\cellcolor[RGB]{242, 242, 242}\huxtpad{6pt + 1em}\raggedright \hspace{6pt} MOOC10/10.1 \hspace{6pt}\huxbpad{6pt}} &
\multicolumn{1}{l!{\huxvb{0, 0, 0}{0}}}{\cellcolor[RGB]{242, 242, 242}\huxtpad{6pt + 1em}\raggedright \hspace{6pt} MOOC10/10.1/2 \hspace{6pt}\huxbpad{6pt}} &
\multicolumn{1}{l!{\huxvb{0, 0, 0}{0}}}{\cellcolor[RGB]{242, 242, 242}\huxtpad{6pt + 1em}\raggedright \hspace{6pt} 1.7 \hspace{6pt}\huxbpad{6pt}} &
\multicolumn{1}{l!{\huxvb{0, 0, 0}{0}}}{\cellcolor[RGB]{242, 242, 242}\huxtpad{6pt + 1em}\raggedright \hspace{6pt} 6 porc:ne boterpotten \hspace{6pt}\huxbpad{6pt}} &
\multicolumn{1}{l!{\huxvb{0, 0, 0}{0.4}}}{\cellcolor[RGB]{242, 242, 242}\huxtpad{6pt + 1em}\raggedright \hspace{6pt} china \hspace{6pt}\huxbpad{6pt}} \tabularnewline[-0.5pt]


\hhline{>{\huxb{0, 0, 0}{0.4}}->{\huxb{0, 0, 0}{0.4}}->{\huxb{0, 0, 0}{0.4}}->{\huxb{0, 0, 0}{0.4}}->{\huxb{0, 0, 0}{0.4}}-}
\arrayrulecolor{black}
\end{tabular}
\end{threeparttable}\par\end{centerbox}

\end{table}
 

\hypertarget{exploratory-data-analysis}{%
\section{Exploratory Data Analysis}\label{exploratory-data-analysis}}

Figure 1 provides a scatter plot of the total auction values between
1720 and 1820. There is a general upwards trend in the value of auctions
over time. However, a large proportion of auctions remain under 50
Rix-dollars (currency at the time). Grouping the goods according to
household, kitchen, farming and select goods allows for the total
auction value to be broken down according to a different collections of
goods present at the auction.

\newpage

\begin{center}\includegraphics{Project_write_up_files/figure-latex/Figure1-1} \end{center}

Figure 2 makes it clear that china, cups and chairs make up a large
amount of the household goods sold per auction. However, the number of
household goods, in absolute terms, does decline over time.

\begin{center}\includegraphics{Project_write_up_files/figure-latex/Figure2-1} \end{center}

Figure 3 shows that utensils and plates make up the largest amount of
kitchen goods sold per auction. As with household goods, kitchen goods
also decline in absolute terms over time.

\begin{center}\includegraphics{Project_write_up_files/figure-latex/Figure3-1} \end{center}

Figure 4 shows that the total number of farming goods is much higher
than that of household and kitchen goods which peak at 200 and 120,
respectively. Cattle and sheep are the most prominent farming goods sold
per auction.

\begin{center}\includegraphics{Project_write_up_files/figure-latex/Figure4-1} \end{center}

Figure 5 shows the average number of art products, books and slaves sold
per auction over time. The absolute number of select goods is lower than
the number of household, kitchen and farming goods. The umber of select
goods per auction is more consistent over time than the other
collections of goods.

\begin{center}\includegraphics{Project_write_up_files/figure-latex/Figure5-1} \end{center}

The figures above indicate that the following goods the most frequently
appear per auction: china, plates, utensils, sheep cattle, books, slaves
and art products. Figure 6 shows box plots for some of these frequently
purchased good types. For visual purposes, the y-axis was capped at 200
but there are a few outliers exceeding this number. Cattle and utensils
appear to be some of the most frequently purchased goods throughout
auctions yet the average number of goods per auction for each type
remains very low.

\begin{center}\includegraphics{Project_write_up_files/figure-latex/Figure6-1} \end{center}

In terms of attendance by titled men and women, Figure 7 shows the
trends over time. There is a downward trend in attendance by titled
purchasers from 1750 onwards and titled men always outnumber titled
women at auctions. Although this univariate analysis identifies
potentially important good types, their relationship with auction total
values is more important.

\begin{center}\includegraphics{Project_write_up_files/figure-latex/Figure7-1} \end{center}

Figure 8 shows how certain good type purchases relate to total auction
values. This shows that the data is heavily clustered close to 0 due to
the small number of tagged goods present at each auction.

\begin{center}\includegraphics{Project_write_up_files/figure-latex/Figure8-1} \end{center}

\hypertarget{target-and-feature-engineering}{%
\section{Target and feature
engineering}\label{target-and-feature-engineering}}

The target in this project is the auction total value. The features used
to predict the target include: the total number of items within a good
type per auction, total number of items within an auction (auction
size), number of titled men and women in attendance per auction and the
year of auction. Figure 9 provides evidence that the features are skewed
left. This is solved by applying a log transformation to all features
(except for the date).

\begin{center}\includegraphics{Project_write_up_files/figure-latex/Figure9-1} \end{center}

\hypertarget{methodology}{%
\section{Methodology}\label{methodology}}

This project aims to predict auction values based on different types of
goods present at an auction. Predicting a continuous, numerical value
presents a regression problem. As a result, the k-Nearest Neighbours
(KNN) model and a random forest model will be used. A linear regression
model will be used to determine how well machine learning models compare
to a regression. The metric used to assess the quality of the machine
learning predictions will be root-mean-squared-error (RMSE). This metric
indicates how far predictions are from actual values on average.
Additionally, r\^{}2 will be used to determine how well the model fits
the data. A default model will be run for each model before conducting
hyperparameter tuning to improve the model.

\hypertarget{knn-model}{%
\section{KNN model}\label{knn-model}}

The KNN model generates predictions for a new observation based on
matching it to the most similar training observations and aggregating
their values. When assessing a new observation (auction), it will
compare the bundle of goods present to other auctions with the most
similar bundles of goods sold and average their total values to
determine a prediction. Hyperparameter tuning is required to set the
optimal value of k. Figure 10 shows that k=5 is the optimal value. New
observations predictions will be the average of its five most similar
auctions.

\begin{center}\includegraphics{Project_write_up_files/figure-latex/Figure10-1} \end{center}

With the training data set, RMSE is 1.35 and r\^{}2 is 0.59. When the
model uses the testing data,however, RMSE reduces to 1.17 and r\^{}2
increases to 0.83. This would indicate that the predictive accuracy and
model fit improved between the training and testing data. It is
important to note that it is a problem that RMSE decreases and r\^{}2
increases when moving from the training to the testing dataset. One
would expect to see the opposite. Possible reasons for this may include
data leakage. After checking for this, I could not come to an
explanation for why this was occurring.The mean logged auction total is
6.32 which means that the average difference between predictions and
actual values is 1.35. As a proportion of the mean total auction value,
this figure is quite large. Figure 11 shows that a very small proportion
of predictions are within 0.3 points of the actual value. This indicates
that knn is a poor model for predicting total auction values.

\begin{center}\includegraphics{Project_write_up_files/figure-latex/Fgiure11-1} \end{center}

\hypertarget{random-forest-model}{%
\section{Random Forest Model}\label{random-forest-model}}

Random forest models include multiple decision trees to make
predictions. Decision trees are built independently on a subset of data
to introduce diversity among trees and reduce overfitting. Additionally,
random features are selected at each node of the tree to cover a wider
breadth of the data and include more robust predictions. Once the
decision trees are trained, predictions of each tree are aggregated to
generate predictions. The randomness of data and feature selection
decreases variance in predictions and aggregating tree-level predictions
minimises bias introduced within trees. Tuning parameters such as the
number of trees within the random forest and the number of features
randomly selected at each node should improve predictions. Furthermore,
the node size can also be tuned as it refers to the minimum number of
observations required at a terminal node to ensure another split.

The default model includes 500 trees, 10 features randomly selected at
each split and a target node size of 10. The RMSE is 1.1 and the r\^{}2
is 0.73. A random forest model offers information on each features'
importance in generating predictions. It is clear from Figure 12 that
the number of slaves at an auction is the most important feature,
followed by the size of the auction, the year the auction took place and
the number of titled men in attendance. These features make sense within
the context of the Cape Colony.

\begin{center}\includegraphics{Project_write_up_files/figure-latex/Figure12, -1} \end{center}

After completing a grid search to tune parameters, the default model is
found to perform the best. When moving from the training to the testing
data, RMSE increases to 2.68. Figure 13 below shows that only a small
proportion of predictions are within 0.1 of the predicted value. Based
on RMSE scores, the knn model performs better than the random forest
model as its RMSE is 1.17.

\begin{center}\includegraphics{Project_write_up_files/figure-latex/Figure13-1} \end{center}

\hypertarget{linear-regression}{%
\section{Linear regression}\label{linear-regression}}

A linear regression is shown below. The RMSE is 1.13 and the r\^{}2 is
0.71. This linear regression model therefore performs better than the
machine learning models with a higher prediction accuracy and better
fit.

\hypertarget{critiques}{%
\section{Critiques}\label{critiques}}

Although the linear regression performed better than the knn and random
forest model, all three models have low predictive accuracy. It is clear
that the subset of goods used to predict total auction values, are
insufficient predictors of total auction values.

\hypertarget{conclusion}{%
\section{Conclusion}\label{conclusion}}

Predicting total auction value using a subset of goods present at
auctions resulted in very inaccurate models. The knn and random forest
models were tested and performed poorly given the constraints presented
by the features chosen. This highlights the importance of choosing
appropriate features when using using machine learning techniques.

\bibliography{Tex/ref}





\end{document}
